% http://tex.stackexchange.com/questions/112580/rescaling-a-math-symbol
\usepackage{scalerel}
\usepackage{mleftright}
\usepackage{dsfont}
\newcommand\scalesym[2]{\hstretch{#1}{\vstretch{#1}{#2}}}

% -- operators
\DeclareMathOperator{\E}{{}\mathbb{E}}
\DeclareMathOperator{\1}{{}\mathds{1}}
\DeclareMathOperator{\DKL}{{}\mathbb{D}_{\text{\scalebox{0.75}{KL}}}}
\DeclareMathOperator*{\argmin}{arg\,min} % * allows typesetting beneath
\DeclareMathOperator*{\argmax}{arg\,max} % * allows typesetting beneath
\DeclareMathOperator*{\maximize}{maximize}
\DeclareMathOperator*{\minimize}{minimize}
\DeclareMathOperator{\ELBO}{\acrshort{ELBO}}
\DeclareMathOperator{\EUBO}{\acrshort{EUBO}}

% -- helpers
\renewcommand{\vec}[1]{\boldsymbol{\mathbf{#1}}}
\renewcommand{\to}{\ensuremath{\rightarrow}}              % add ensure math
\newcommand{\from}{\ensuremath{\leftarrow}}
\newcommand{\given}{\mid}
\renewcommand{\~}{\,\scalesym{0.7}{\thicksim}\,}          % what is this renewing?
\newcommand{\grpP}[1]{\ensuremath{\mleft( #1 \mright)}}   % parens   ()
\newcommand{\grpB}[1]{\ensuremath{\mleft\{ #1 \mright\}}} % braces   {}
\newcommand{\grpS}[1]{\ensuremath{\mleft[ #1 \mright]}}   % brackets []
\newcommand{\fnP}[2]{\ensuremath{#1 \grpP{#2}}}           % parens   ()
\newcommand{\fnB}[2]{\ensuremath{#1 \grpB{#2}}}           % braces   {}
\newcommand{\fnS}[2]{\ensuremath{#1 \grpS{#2}}}           % brackets []
\newcommand{\KL}[2]{\fnP{\DKL}{#1 \,\|\; #2}}             % kl divergence
\newcommand{\Ex}[2][]{\fnS{\E_{#1}}{#2}}                  % expectation

% -- special
\newcommand{\p}[2][]{\fnP{p_{#1}}{#2}}
\newcommand{\q}[2][]{\fnP{q_{#1}}{#2}}

%%% Local Variables:
%%% mode: latex
%%% TeX-master: "notes"
%%% End:
